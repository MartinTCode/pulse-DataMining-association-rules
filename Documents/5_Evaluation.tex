\label{chap:evaluation}

The results of our association rule mining were evaluated by examining both the statistical 
measures and the practical implications of the rules. When experimenting with different support 
thresholds, we observed that a support of 0.1 produced only a handful of rules, most of which 
were too obvious to be useful. On the other hand, lowering the support to 0.001 led to a flood 
of trivial rules where nearly every weak co-occurrence was counted. A more balanced setting 
was achieved with a support of 0.01 combined with a confidence of 0.6, which yielded a reasonable 
number of interpretable rules.

Among these, the most consistent and interesting rule was the relation between Toast and Coffee. 
Both RapidMiner and Knime identified this pattern: if a customer buys Toast, there is a 
70.4\% chance that they will also buy Coffee. The rule had a support of 0.024, meaning that 
2.4\% of all transactions contained this combination. While Coffee appeared in many rules due to 
its overall popularity in the dataset, the repeated identification of Toast as a strong antecedent 
gave this rule particular credibility. It demonstrates a clear and actionable relationship that 
the bakery could use for bundle offers or product placement.

Other rules, such as the combinations of Cake with Hot Chocolate leading to Coffee, or Salad 
leading to Coffee, were also detected. These highlight more subtle customer behaviours, although 
some of them are less intuitive. For example, Salad being linked with Coffee may reflect that 
customers who buy a light meal often also add a beverage. However, rules involving very common 
items like Coffee and Bread tend to dominate the analysis, which makes it difficult to highlight 
more unexpected patterns. This limitation is common in market basket analysis and underscores 
the need for careful interpretation.

An important observation during the evaluation was the difference between the two tools. RapidMiner 
tended to generate a larger set of rules compared to Knime under identical settings, while 
Knime was stricter and only reproduced the strongest patterns. Despite these differences, the 
tools consistently agreed on the key finding, which increases confidence in the robustness of 
the results.

Although association rules are valuable for identifying co-occurrences, it is also important to 
recognise their limitations. They do not explain why two items appear together and cannot 
establish causation. Nevertheless, in the context of a bakery, the discovered rules still provide 
relevant and actionable insights that can guide decisions about marketing campaigns, store layout, 
and product bundling.

