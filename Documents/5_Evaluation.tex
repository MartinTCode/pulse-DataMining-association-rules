\label{chap:evaluation}

When experimenting with different support thresholds, we observed that a support of 0.1 produced only a handful of 
rules, most of which were too obvious to be useful. On the other hand, lowering the support to 0.001 led to a flood 
of trivial rules where nearly every weak co-occurrence was counted. A more balanced setting was achieved with a 
support of 0.01 combined with a confidence of 0.6, which yielded a reasonable number of interpretable rules. This 
is consistent with Tan et al.~\cite[Ch.~4]{courseLitt}, who emphasize that support and confidence must be tuned to 
balance coverage and usefulness.  

Among these, the most consistent and interesting rule was the relation between Toast and Coffee. Both RapidMiner 
and Knime identified this pattern: if a customer buys Toast, there is a 70.4\% chance that they will also buy 
Coffee. The rule had a support of 0.024, meaning that 2.4\% of all transactions contained this combination. While 
Coffee appeared in many rules due to its overall popularity in the dataset, the repeated identification of Toast 
as a strong antecedent gave this rule particular credibility. It demonstrates a clear and actionable relationship 
that the bakery could use for bundle offers or product placement. Similar dominance of popular items is also 
discussed by Tan et al.~\cite[Ch.~4]{courseLitt}, who caution that highly frequent items often drive many rules but 
may not always provide new insight.  

Other rules, such as the combinations of Cake with Hot Chocolate leading to Coffee, or Salad leading to Coffee, 
were also detected. These highlight more subtle customer behaviours, although some of them are less intuitive. 
For example, Salad being linked with Coffee may reflect that customers who buy a light meal often also add a 
beverage. However, rules involving very common items like Coffee and Bread tend to dominate the analysis, which 
makes it difficult to highlight more unexpected patterns. This limitation is common in market basket analysis and 
is described by Tan et al.~\cite[Ch.~4]{courseLitt} as a key challenge in finding “interesting” patterns beyond 
frequency-based rules.  

Although association rules are valuable for identifying co-occurrences, it is also important to recognise their 
limitations. They do not explain why two items appear together and cannot establish causation. As Tan et 
al.~\cite[Ch.~4]{courseLitt} note, association rules describe correlation, not causality. Nevertheless, in the 
context of a bakery, the discovered rules still provide relevant and actionable insights that can guide decisions 
about marketing campaigns, store layout, and product bundling.  

