\label{chap:data-understanding}

We are working with the \textit{Bakery Transactions} dataset from Kaggle~\cite{bakerydata}, which contains detailed point-of-sale records from a bakery between January 2016 and December 2017. 
Each row represents an item purchased in a transaction. 
In total, the dataset includes around 20,500 records and meets the requirement of having more than 1,000 rows for association rule mining.  

\subsection*{Data type}
The Bakery Transactions dataset is an example of \textbf{structured data}, 
since it is stored in a tabular CSV format with clearly defined attributes such as 
TransactionNo, Items, DateTime, Daypart, and DayType.  
Most of the attributes are categorical (e.g., Items, Daypart, DayType), 
while DateTime is temporal.  
There are no numerical features of direct analytical interest, aside from identifiers 
or indices such as TransactionNo.  
This makes the dataset particularly well-suited for association rule mining, 
an \textbf{unsupervised learning} technique that requires categorical transaction 
data to identify frequent itemsets and co-occurrence patterns.
% TODO: Add source from book. 

\subsection*{Main attributes in the dataset}
\begin{itemize}
    \item \textbf{TransactionNo:} A unique identifier for each transaction.  
    Range: 2,552 to 9,684.  
    Mean: $\sim$4,980 with a standard deviation of $\sim$2,800.  
    No missing or mismatched values.  

    \item \textbf{Items:} The products purchased in each transaction.  
    There are 94 unique items.  
    The most common product is \textit{Coffee} (27\%), followed by \textit{Bread} (16\%).  

    \item \textbf{DateTime:} The timestamp of each transaction.  
    Range: 11 Jan 2016 to 3 Dec 2017.  
    % TODO: Add explanation about range differences in DateTime once calculated
    \textbf{FILL IN range differences!!!! CANT BE READ RELIABLY FROM SOURCE GRAPH.}
    No missing values.  

    \item \textbf{Daypart:} Indicates the part of the day when the transaction occurred (Morning, Afternoon, Evening, Night).  
    Most transactions occurred in the Afternoon (56\%) and Morning (41\%).  
    Very few happened in the evening or night, which makes sense for a bakery.  

    \item \textbf{DayType:} Classifies each transaction as a Weekday (62\%) or Weekend (38\%).  
    At first glance this suggests that most purchases happen during the workweek. 
    However, since weekdays cover five days and weekends only two, the comparison is not direct

\end{itemize}

\subsection*{Initial observations}
The dataset is clean, with no missing or mismatched values.  
Item names are already standardized (e.g., ``Coffee'' is consistently spelled), 
so only minimal preprocessing will be needed.  
Coffee and bread are the most common items, 
which we expect to appear strongly in the association rules.  

The temporal attributes also add context: most transactions take place in the morning and afternoon, which fits the business of a bakery.  
The weekday/weekend split (62\% vs.\ 38\%) should be interpreted with caution, since weekdays cover five days while weekends only two.  
If we average these proportions over the number of days, this corresponds to roughly $62 \div 5 \approx 12.4\%$ of weekly transactions per weekday and $38 \div 2 \approx 19.0\%$ per weekend day.  
This suggests that weekends might have relatively higher activity on a per-day basis, although this is only an initial observation.  
A more detailed analysis would be needed to confirm this, since transaction volume may not be evenly distributed across weekdays (e.g., Mondays could have many more transactions than other weekdays).
 

Overall, the dataset is well-suited for market basket analysis.  
It not only allows us to study which items are bought together but also provides additional 
information about when purchases occur 
(time of day, weekday/weekend), which could be explored in further analysis.
