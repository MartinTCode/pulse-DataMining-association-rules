\label{chap:data-understanding}

We are working with the \textit{Bakery Transactions} dataset from Kaggle~\cite{bakerydata}, 
which contains detailed point-of-sale records from a bakery between January 2016 and December 2017. 
Each row represents a single item purchased, and multiple rows can belong to the same transaction.  
In total, the dataset includes around 20,500 records and meets the requirement of having more than 
1,000 rows for association rule mining.
A copy of the dataset is also available via Dropbox~\cite{csvdataset}.  

\subsection*{Data type}
The Bakery Transactions dataset is \textbf{structured}, 
stored in CSV format where each row corresponds to an item purchase, 
linked to a transaction via the \texttt{TransactionNo} attribute.  
Multiple rows can share the same transaction identifier, meaning that a single 
basket may contain several products.  
Basket sizes range from 1 to 11 items. 
According to Tan et al.~\cite[Ch.~2, Sec.~2.1.2]{courseLitt}, 
this corresponds to \textit{transaction data} (market basket data), 
a special type of record data.  
Most attributes are categorical, consistent with the definition of nominal 
attributes~\cite[Ch.~2, Sec.~2.1.1]{courseLitt}.  
This structure makes the dataset well-suited for association rule mining, 
an \textbf{unsupervised learning} technique that discovers frequent itemsets 
and co-occurrence patterns in transaction data~\cite[Ch.~5, Sec.~5.1]{courseLitt}.


\subsection*{Main attributes in the dataset}
\begin{itemize}
    \item \textbf{TransactionNo:} An identifier that groups together all items in the same basket (transaction).  
    Basket sizes range from 1 to 11 items.  
    Range: 2,552 to 9,684.  
    Mean: $\sim$4,980 with a standard deviation of $\sim$2,800.  
    No missing or mismatched values.  

    \item \textbf{Items:} The products purchased in each transaction.  
There are 94 unique items, consistently capitalized and without duplicate spellings.  
However, some categories overlap semantically (e.g., ``Sandwich'' vs.\ ``Chicken sand''), 
and several products are expressed as multi-word names.  
The most common product is \textit{Coffee} (27\%), followed by \textit{Bread} (16\%).  

    \item \textbf{DateTime:} The timestamp of each transaction.  
    Range: 11 Jan 2016 to 3 Dec 2017.  
    No missing values.  
    Transaction volume was not evenly distributed: the period from 
    November 2016 to March 2017 concentrated nearly 64\% of all orders, 
    while the surrounding months (Jan--Oct 2016 and Apr--Dec 2017) 
    accounted for less than 36\% combined.  
    This strong seasonality indicates that the bakery experienced 
    a pronounced high-demand span followed by lower activity.  

    \item \textbf{Daypart:} Indicates the part of the day when the transaction occurred (Morning, Afternoon, Evening, Night).  
    Most transactions occurred in the Afternoon (56\%) and Morning (41\%).  
    Very few happened in the evening or night, which makes sense for a bakery.  

    \item \textbf{DayType:} Classifies each transaction as a Weekday (62\%) or Weekend (38\%).  
    At first glance this suggests that most purchases happen during the workweek. 
    However, since weekdays cover five days and weekends only two, the comparison is not direct

\end{itemize}

\subsection*{Initial observations}
The exploratory graphs provided in the Kaggle source~\cite{bakerydata} 
support these observations (see dataset column views for visual summaries).  
Our percentages and seasonality measures are the result of additional analysis 
performed on the raw data, beyond what is directly shown in the source.  


The dataset is clean, with no missing or mismatched values.  
No noise, outliers, or duplicate records were observed.  
Item names are fairly standardized (94 unique values, consistently capitalized), 
though a few categories overlap semantically (e.g., ``Sandwich'' vs.\ ``Chicken sand'').  
Coffee and bread are the most common items, 
which we expect to appear strongly in the association rules.  

The temporal attributes reveal important context.  
Most transactions occurred in the morning and afternoon, which aligns with 
typical bakery operations.  
The weekday/weekend split (62\% vs.\ 38\%) must be interpreted with care: 
on a per-day basis, weekends represent a larger share 
($\sim$19\% per day vs.\ $\sim$12\% per weekday).  
More notably, transaction volume shows strong seasonality: 
a pronounced peak between November 2016 and March 2017 
concentrated almost two-thirds of all orders, 
while the rest of the months together accounted for just over one-third.  
This suggests external factors such as holidays, local conditions, 
or promotions likely influenced customer behavior.  

Overall, the dataset is well-suited for market basket analysis.  
It not only enables the discovery of co-purchase patterns 
but also provides temporal context (time of day, weekday/weekend, 
and seasonal variation), which could be leveraged in further analysis. 

These patterns can provide the basis for identifying associations that can inform 
the bakery's marketing, layout, and inventory planning decisions 
(see \nameref{chap:business-understanding}).
