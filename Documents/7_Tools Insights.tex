\label{chap:tool-insights}

In the previous assignment, we noted that the learning curve can be steep without prior experience
in data mining, its vocabulary, or the tools themselves, and this still applies here. The  online
resources provided by RapidMiner and Knime is helpful, as well as built-in documentation that
explains the functionality and application of different operators and nodes. From a practical
perspective, now that we have become a bit more familiar with the tools, they are somewhat similar
even though the naming and, to some extent, the functionality of their operators/nodes differ.

For the structured input data in this case, very little preprocessing was required. However, it is
easy to make processes more complicated than necessary, since there are often multiple ways to
achieve the same result. Configuring an operator or node can sometimes be challenging, but this is
mainly due to being beginners in both data mining and in using the tools for it.

For association rule mining, RapidMiner requires the use of the FP-Growth and Create Association Rules
operators, whereas Knime can perform the same task with only the Association Rule Learning node. However,
the results are not identical, which suggests that the two tools do not implement the technique in exactly
the same way.