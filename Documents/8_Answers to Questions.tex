\label{chap:answer-to-questions}

\section*{What rules did you find?}
The association rules discovered in our analysis showed several consistent patterns. The strongest and most 
reliable rule was the link between Toast and Coffee, with a confidence of over 70\% and a support of 2.4\%. 
This means that whenever Toast was purchased, Coffee was very likely to be present in the same basket. Both 
RapidMiner and Knime arrived at this rule, even though RapidMiner also produced a larger set of additional 
rules under the same parameter configuration. These additional rules included combinations such as Cake with 
Hot Chocolate leading to Coffee, Salad leading to Coffee, and the product Keeping It Local leading to 
Coffee with high confidence.


\section*{What attributes are most strongly associated?}
Coffee dominated as the consequent in most rules, which is not surprising given that it is the most common 
product in the dataset. What makes the association with Toast stand out is that it was found consistently 
across both RapidMiner and Knime with strong support and confidence. This repeated detection suggests that 
the relationship is not random but rather reflects a genuine purchasing habit. Other attributes such as Cake, 
Hot Chocolate, and Salad also appeared in rules, but only when combined with other items. Their associations 
were less frequent and not as robust. Overall, the strongest and most reliable association was between Toast 
and Coffee, making these two attributes the most strongly linked in the dataset.


\section*{Are there products frequently connected that surprise you? Why?}
Some associations were less expected, particularly the link between Salad and Coffee. At first glance, these 
items do not seem like a natural pair, but the rule may reflect that customers who purchase a lighter meal 
often add a hot drink to complete their order. Another interesting result was the combination of Cake and Hot 
Chocolate leading to Coffee. Even when a beverage was already included in the basket, customers sometimes still 
purchased Coffee, possibly because they were buying for multiple people or preferred to have both options. 
These patterns highlight how association rule mining can uncover connections that are not immediately intuitive 
and may not be predicted without a data-driven approach.


\section*{How much did you have to test different support and confidence values before finding useful rules?}
Finding useful rules required several iterations with different support and confidence thresholds. At high 
support values the ruleset was too small and contained only trivial relations, while at very low support 
levels the ruleset became too large and unfocused. After testing three different support thresholds and 
adjusting the confidence range, we settled on the combination that produced a manageable number of meaningful 
rules. This process reflects the trial-and-error nature of association rule mining, where parameters must be 
tuned to balance coverage and interpretability.


\section*{Were any of your rules good enough to base decisions on? Why or why not?}
In terms of business value, some rules were strong enough to guide decisions. The Toast and Coffee rule, for 
example, could directly motivate bundle deals or adjustments in product placement. Other rules, while statistically 
valid, mainly confirmed what would already be expected from everyday experience at a bakery. Even so, such 
confirmations are valuable as they provide evidence that the dataset reflects realistic purchasing behaviour. 
Together, the rules form a basis for recommendations that the bakery could consider implementing in practice.